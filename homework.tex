\documentclass{article}
\usepackage{graphicx} 
\usepackage{theJackPack}

\title{Homework 4 - Math 525}
\author{Jack Brolin}
\date{July 2023}

\begin{document}

\maketitle

\begin{jacklist}
    \begin{framed} 
    \item [\textbf{P. 1}] Consider the simplex method applied to a standard form problem and assume that the rows of the matrix 
        $A$ are linearly independent. For each of the statements that follow, give either a proof or a counterexample.
        \begin{itemize}
            \item [a.] An iteration of the simplex method may move the feasible solution by a positive distance while leaving 
                the cost unchanged.
            \item [b.] A variable that has just left the basis cannot reenter in the very next iteration.
            \item [c.] A variable that has just entered the basis cannot leave in the very next iteration.
            \item [d.] If there is a nondegenerate optimal basis, then there exists a unique optimal basis.
            \item [e.] If $x$ is an optimal solution found by the simplex method, no more than $m$ of its components can be positive, 
                where $m$ is the number of equality constraints.
        \end{itemize}
    \end{framed}
    \begin{itemize}
        \item [a.] False. $c^\prime x < c^\prime(x + \theta d)$ given $\theta > 0$ and $d > 0$, values that are always 
            present in any positive direction. Thus the cost cannot be unchanged.
        \item [b.] True. When a variable leaves the basis, its corresponding coefficient becomes 0. Since pivoting is done strictly 
            negative values, and the pivoting column corresponds to variables entering the basis, it is not possible for a variable to 
            exit and then enter the next iteration. 
        \item [c.]
        \item [d.] False. We can find a LP problem with multiple, non-degenerate, optimal solutions. Consider the LP problem 
            \begin{align*}
                \text{max } & x_1 + x_2 \\
                \text{s.t. } & x_1 + x_2 \leq 1 \\
                            &x \geq 0
            \end{align*} 
            Now every point where $x_1 + x_2 = 1$ and $x \geq 0$ is optimal. 
        \item [e.] True. By Theorem 2.4(b), $x_i = 0 \forall i \neq B(1),\ldots,B(m)$, so there can be only $m$ positive variables. 
    \end{itemize}
\newpage
    \begin{framed} 
    \item [\textbf{P. 3}] Solve completely (i.e., both Phase I and Phase II) via the simplex method the following problem: 
    \[ 
        \begin{array}{rc}
        \text { minimize } & 2 x_{1}+3 x_{2}+3 x_{3}+x_{4}-2 x_{5} \\
        \text { subject to } & x_{1}+3 x_{2}+4 x_{4}+x_{5}=2 \\
        & x_{1}+2 x_{2}-3 x_{4}+x_{5}=2 \\
        & -x_{1}-4 x_{2}+3 x_{3}=1 \\
        & x_{1}, x_{2}, x_{3}, x_{4}, x_{5} \geq 0
        \end{array}
    \] 
    \end{framed}
    We see 4 iterations of Phase I. We construct the first tableau with the introduction of auxiliary variables: 
    \begin{center}
        \begin{tabular}{|c|cccccccc|}
            \hline
            -5&-1&-1&-3&-1&-2&0&0&0\\
            \hline
            2&1&3&0&4&1&1&0&0\\
            2&1&2&0&-3&1&0&1&0\\
            1&-1&-4&3&0&0&0&0&1\\
            \hline
        \end{tabular}
    \end{center}
    We get 3 as the pivot variable with $x_3$ entering the basis and $x_6$ exiting. Performing the necessary operations we get:
    \begin{center}
        \begin{tabular}{|c|ccccccc|}
            \hline
            -4&-2&-5&0&1&-2&0&1\\
            \hline
            2&1&3&0&4&1&0&0\\
            2&1&2&0&-3&1&1&0\\
            $\rfrac{1}{3}$&$-\rfrac{1}{3}$&$-\rfrac{4}{3}$&1&0&0&0&$\rfrac{1}{3}$ \\
            \hline
        \end{tabular}
    \end{center}
    Note that we completely removed $x_6$ from the tableau as it no longer serves any purpose. We now let $x_2$ enter the basis with
    $x_7$ exiting: 
    \begin{center}
        \begin{tabular}{|c|cccccc|}
            \hline
            $-\rfrac{2}{3}$&$-\rfrac{1}{3}$&0&0&$\rfrac{17}{3}$&$-\rfrac{1}{3}$&0\\
            \hline
            $\rfrac{2}{3}$&$\rfrac{1}{3}$&1&0&$\rfrac{4}{3}$&$\rfrac{1}{3}$&0\\
            $\rfrac{2}{3}$&$\rfrac{1}{3}$&0&0&$-\rfrac{17}{3}$&$\rfrac{1}{3}$&1\\
            $\rfrac{11}{9}$&$\rfrac{1}{9}$&0&1&$\rfrac{16}{9}$&$\rfrac{4}{9}$&0\\
            \hline
        \end{tabular}
    \end{center}
    Now we let $x_1$ enter with $x_7$ exiting. Thus we get the final tableau: 
    \begin{center}
        \begin{tabular}{|c|ccccc|}
            \hline
            0&0&0&0&0&0\\
            \hline
            0&0&1&0&0&0\\
            2&1&0&0&-17&1\\
            1&0&0&1&$\rfrac{11}{3}$&$\rfrac{1}{3}$\\
            \hline
        \end{tabular}
    \end{center}
    and the basic feasible solution is $x = (2,0,1,0,0)$. \\
    \textbf{Phase II} \\
    We copy over the same tableau and calculate the new reduced cost and cost to get :
    \begin{center}
        \begin{tabular}{|c|ccccc|}
            \hline
            7&2&3&3&-2&3\\
            \hline
            2&1&0&0&-17&0\\
            0&0&1&0&7&0\\
            1&0&0&1&$\rfrac{11}{3}$&$\rfrac{1}{3}$\\
            \hline
        \end{tabular}
    \end{center}
    We do one more iteration to get:
    \begin{center}
        \begin{tabular}{|c|ccccc|}
            \hline
            7&2&$\rfrac{23}{7}$&3&0&3\\
            \hline
            2&1&&0&0&1\\
            0&0&&0&1&0\\
            1&0&&1&0&0\\
            \hline
        \end{tabular}
    \end{center}
    We now see that there are no more negative values in $c$ so the optimal solution is $x = (2,0,1,0,0)$. Note that we can leave some
    values blank as calculating them does not give any information useful to the problem. 
\newpage
    \begin{framed} 
    \item [\textbf{P. 4}] While solving a standard form problem, we arrive at the following tableau, with $x_{3}, x_{4}$, and $x_{5}$ 
        being the basic variables: 
        \begin{center}
        \begin{tabular}{|c|ccccc|}
            \hline-10 & $\delta$ & -2 & 0 & 0 & 0 \\
            \hline 4 & -1 & $\eta$ & 1 & 0 & 0 \\
            1 & $\alpha$ & -4 & 0 & 1 & 0 \\
            $\beta$ & $\gamma$ & 3 & 0 & 0 & 1 \\
            \hline
        \end{tabular}
        \end{center}
        The entries $\alpha, \beta, \gamma, \delta, \eta$ in the tableau are unknown parameters. 
        For each one of the following statements, find some parameter values that will make the statement true:
        \begin{itemize}
            \item [a.] The current solution is optimal and there are multiple optimal solutions.
            \item [b.] The optimal cost is $-\infty$.
            \item [c.] The current solution is feasible but not optimal.
        \end{itemize}
    \end{framed}
    \begin{itemize}
        \item [a.]  
        \item [b.] Note that $\beta \geq 0$ for $x$ to be feasible. We then have to find some $u$ such that no component is positive. 
            This, as seem from section 3.2, ensures that the optimal cost is $-\infty$. Thus, we choose $\delta \leq 0$ and 
            $\alpha, \gamma < 0$. $\eta$ is free. A possible tableau could be: 
            \begin{center}
                \begin{tabular}{|c|ccccc|}
                    \hline
                    -10&-1&-2&0&0&0\\
                    \hline
                    4&-1&7&1&0&0\\
                    1&-1&-4&0&1&0\\
                    1&-1&3&0&0&1\\
                    \hline
                \end{tabular}
            \end{center}    
        \item [c.] To be feasible $\beta \geq 0$. The rest of the variables are free as long as at least one 
            $\delta, \gamma, \alpha \geq 0$ to avoid an optimal cost of $-\infty$. We will require $\alpha, \beta, \gamma, \delta, 
            \eta > 0$ giving a possible tableau of:
            \begin{center}
                \begin{tabular}{|c|ccccc|}
                    \hline
                    -10&5&-2&0&0&0\\
                    \hline
                    4&-1&1&1&0&0\\
                    1&1&-4&0&1&0\\
                    5&1&3&0&0&1\\
                    \hline
                \end{tabular}
            \end{center}
    \end{itemize}
\newpage
    \begin{framed} 
    \item [\textbf{P. 6}] Consider the following linear programming problem with a single constraint: 
    \[ 
        \begin{aligned}
            \text { minimize } & \sum_{i=1}^{n} c_{i} x_{i} \\
            \text { subject to } & \sum_{i=1}^{n} a_{i} x_{i}=b \\
            & x_{i} \geq 0, \quad i=1, \ldots, n .
        \end{aligned}
    \] 
    \begin{itemize}
        \item [a.] Derive a simple test for checking the feasibility of this problem. (Hint: Discuss when $b=0, b>0$ and $b<0)$ 
        \item [b.] Assuming that the optimal cost is finite, develop a simple method for obtaining an optimal solution directly.
    \end{itemize}
    \end{framed}
    \begin{itemize}
        \item [a.] As instructed, we break $b$ into 3 cases: \\
            \begin{itemize}
                \item $b = 0:$ This is trivial as $x = \textbf{0}$ is always a solution 
                \item $b > 0:$ We only require some $a_j > 0$ as we can choose $x$ such that $x_i = 0 \quad \forall i \neq j$ and 
                    $x_j = \frac{b}{a_j}$. 
                \item $b < 0:$ Similarly, this is feasible of some $a_j < 0$. Again, we can always choose some $x$ such that
                    $x_i = 0 \forall i \neq j$ and $x_j = \frac{b}{a_j}$. 
            \end{itemize}
        \item [b.] Noticing that we only have one constraint, we note that the solution will only have 1 non-zero value. To 
            find this value, consider $ \frac{b}{a_i}$. The smallest of these ratios is the optimal cost if we let all other values
            be 0 and $a = \textbf{1}$. Because $a \neq \textbf{1}$ in most cases, we account for this by multiplying $c_i$ to the ratio 
            and considering the smallest. That is, $ \frac{b \cdot c_i}{a_i}$. The smallest of this gives an optimal solution. That is, 
            $x_i = \frac{b}{a_j}$ and $x_j = 0 \forall j \neq i$ where $i$ is the smallest of the $ \frac{b \cdot c_i}{a_i}$. 
    \end{itemize}
\newpage
    \begin{framed} 
    \item [\textbf{P. 8}] Consider the following optimization problem $(P)$ : find a vector $x \in \mathbb{R}^{n}$ that satisfies 
        $A x=0$ and $x \geq 0$, and such that the number of positive components of $x$ is maximized. Let $\left(P^{\prime}\right)$ 
        be the linear program defined as: 
        \[ 
            \begin{aligned}
                \text { maximize } & \sum_{i=1}^{n} y_{i} \\
                \text { subject to } & A(z+y)=0 \\
                & z, y \geq 0 \\
                & y_{i} \leq 1, \quad i=1, \ldots, n .
            \end{aligned}
        \] Show that $\left(P^{\prime}\right)$ can be used to solve $(P)$. (Hint: You can show that $(P)$ and $\left(P^{\prime}\right)$ 
        are equivalent - you must specify how to map a feasible solution of $(P)$ to a feasible solution of 
        $\left(P^{\prime}\right)$ with greater or equal cost, and viceversa.)
    \end{framed}
    \begin{proof}
        Let $x$ be an feasible solution of $P$. Then for any positive multiple of $x$ is also feasible so scale $x$ to $x^\star$ so
        the smallest positive value is greater than 1. Let another variable, $y$ have components $y_i = 1$ if $x^\star > 0$ and $y_i = 
        0$. Otherwise, let $z = x^\star -y$ and note that $z$ is feasible. We can use $P$ since we can force that $y_i = 0$ or $1$ 
        for all $i$. Letting $z = x^star - y$, we can maximize $\sum_{i=1}^ny_i$ where the result just gives us the number of positive 
        components. 
    \end{proof}
\end{jacklist}
\end{document}
